\documentclass{article}
\usepackage{pdfpages}
\usepackage[margin=10pt,paperwidth=12in,paperheight=12in]{geometry}
\usepackage[many]{tcolorbox}
\usepackage{xepersian}
\settextfont{Noto Naskh Arabic}
\setlatintextfont{Noto Sans}
\begin{document}

\newcommand{\note}[1]{
  \begin{tcolorbox}[width=0.3\linewidth]
    \begin{RTL}
      #1
    \end{RTL}
  \end{tcolorbox}
}

\includepdf[pages=1,offset=150 0,frame=true,fitpaper=false,pagecommand={
    \vspace*{5cm}
    \begin{tcolorbox}[width=0.3\linewidth]
      \begin{RTL}
        علم العربیة: مضاف و مضاف الیه است و در واقع علم اللغة العربیة بوده و موصوف که اللغة بوده حذف شده است. 
        گاهی اوقات موصوف را در لغت نمی‌آورند.
      \end{RTL}
    \end{tcolorbox}
}]{page6_7.pdf}


\end{document}
