\documentclass{article}
\usepackage{pdfpages}
\usepackage{environ}
\usepackage[margin=10pt,paperwidth=12in,paperheight=12in]{geometry}
\usepackage[many]{tcolorbox}
\usepackage{xepersian}
\settextfont{Noto Naskh Arabic}
\setlatintextfont{Noto Sans}

\def\labelitemi{$-$}

\NewEnviron{addpage}[1]{
  \includepdf[pages=#1,offset=150 0,frame=true,fitpaper=false,pagecommand={
      \BODY
  }]{book.pdf}
}

\NewEnviron{note}{
  \begin{tcolorbox}[width=0.3\linewidth]
    \begin{RTL}
      \BODY
    \end{RTL}
  \end{tcolorbox}
}

\begin{document}
\begin{addpage}{1}
  \vspace*{6cm}
  \begin{note}
    علم العربیة: مضاف و مضاف الیه است و در واقع علم اللغة العربیة بوده و موصوف که اللغة بوده حذف شده است. 
    گاهی اوقات موصوف را در لغت نمی‌آورند.
  \end{note}
  \begin{note}
    صناعة: دانش و فن.
    صوغ: ریختن، اسم فاعل آن صائغ است به معنای ریخته‌گر و اسم مفعول آن مصوغ به معنای ریخته شده.
  \end{note}
  \begin{note}
    خبر «انّ» بر آن مقدم نمی‌شود ولی خبر «کان» مقدم می‌شود.
  \end{note}
  \vspace*{4cm}
  \begin{note}
    نظرات دیگر مانند ۲۸ حرف و ۳۰ حرف هم برای تعداد حروف الفبا وجود دارد.
  \end{note}
  \begin{note}
    نشانه‌های اسم:
    \begin{itemize}
    \item مجرور شدن
    \item ال گرفتن
    \item تنوین
    \item منادا واقع شدن
    \item مسند الیه واقع شدن
    \end{itemize}
    
    شعر الفیه:

    بالجرّ و التنوين و النّدا و آل
    
    و مسند للاسم تمييز حصل


  \end{note}
\end{addpage}

\begin{addpage}{2}
  \vspace*{0.3cm}
  \begin{note}
    توطئه: مصدر باب تفعیل است از ریشه و ط ء.
    به معنای کوبیدن و فشار دادن و پستی و بلندی‌ها را گرفتن. مثلا وقتی جاده را صاف می‌کنند. معنای زمینه چینی هم میدهد. در عربی بر خلاف فارسی معنای بدی ندارد.
  \end{note}
  \begin{note}
    الف متحرک همان همزه است، مانند «الاحد». اما الف ساکن آن است که در شمارش حروف الفباء قبل از یاء می‌آید. سه حرف آخر حروف الفباء
    واو، الف، یاء. به آن «لام الف» هم می‌گویند زیرا برای تلفظ آن نیاز به یک «ال» هستیم.
  \end{note}
  \begin{note}
    فی سرد الحروف: لیست کردن حروف.
  \end{note}
  \begin{note}
    علة: به معنای مرض و بیماری است. مریض در معرض تغییر است، مثلا رنگ چهره‌اش تغییر می‌کند. این حروف نیز چون تغییر پذیر هستند حروف عله نام دارند.
    سه حرف آخر حروف الفبا حروف عله هستند، واو، الف، یاء.
  \end{note}
  \begin{note}
    حرکات اصواتی هستند که توسط آن‌ها می‌توانیم یک حرف را ادا کنیم. برخی به آن‌ها مصوت هم می‌گویند. مصوت بلند و مصوت کوتاه داریم. مصوت‌های ضمه و فتحه و کسره هستند.
    مصوت‌های بلند متناظر آنها واو و الف و یاء هستند.

    در جایی که حروف عله حذف می‌شوند مصوت بلند به مصوت کوتاه تبدیل می‌شود.
  \end{note}
  \begin{note}
    بعد از «ما خلاء» کلمه منصوب می‌شود. در واقع کلمه بعدی مفعول‌به است برای «خلاء».
    
    تنها حرفی که هیچ حرکتی بر نمیدارد حرف الف است که همیشه ساکن است.
  \end{note}
  \begin{note}
    اگر کلمه‌ای آخرش واو یا یاء باشد و ما قبل آن ساکن باشد، به آن کلمه «جاری مجرای صحیح» گفته می‌شود، یعنی مشابه و نظیر صحیح.

    اگر در انتهای فعل مضارع واو و یاء باشد، فعل مضارع ناقص، ضمه را نمی‌پذیرد و به جای ضمه از فتحه استفاده می‌کنیم.

    اسم منقوص: اسمی که آخرش یاء باشد و حرف ما قبل یاء هم مکسور باشد.
  \end{note}
  \begin{note}
    به حرفی که بعد از آن حرف عله باشد، حرف «لین» می‌گویند. در صورتی که حرکت حرف لین با حرف عله سازگار باشد، به آن حرف مد هم می‌گویند.
  \end{note}
  \begin{note}
    حرف الف همیشه حرف مد است. زیرا همیشه قبل از آن فتحه است. اگر قبل از حرف الف ضمه یا کسره باشد، به واو و یاء تبدیل میشود و اصلا حرف الف نیست دیگر.
  \end{note}
  \vspace*{2cm}
  \begin{note}
    فقط برخی از اسماء معرب تنوین می‌گیرند و همه آن‌ها تنوین قبول نمی‌کنند.
  \end{note}
\end{addpage}

\begin{addpage}{3}
  \vspace*{5cm}
  \begin{note}
    تشدید ادغام شدن دو حرف یک جنس و یا متقارب است. حروف متقارب مثل نون ساکن و واو، یا نون ساکن و لام، در «یکن لهو» که نون تلفظ نمی‌شود و لام مشدد می‌شود.
  \end{note}
  \begin{note}
    همزه وصل در ابتدای کلام تلفظ می‌شود ولی در اثناء کلام تلفظ نمی‌شود. ولی همیشه نوشته می‌شود.
  \end{note}
  \begin{note}
    همزه قطع هم نوشته می‌شود و هم خوانده می‌شود چه در ابتدای کلام چه در میان کلام.
  \end{note}
  \begin{note}
    همزه باب افعال همزه قطع است.
  \end{note}
  \begin{note}
    ثلاثی مجرد و باقی مزیدها غیر باب افعال همه همزه وصل هستند. رباعی‌ها هم همه همزه وصل هستند. تنها بابی که همه قطع است باب افعال است.
  \end{note}

\end{addpage}

\begin{addpage}{4}
  \vspace*{10cm}
  \begin{note}
    این اعداد که میگوئیم صفت برای موصوف مقدر است، الخامس در واقع بوده السوال الخامس.
  \end{note}
  \begin{note}
    به صورت‌های مختلف صیغ و یا بنا هم می‌گویند. منظور از صور حالاتی است که کلمات با آن متفاوت می‌شوند، مثل فاعل و مفعول. مثلا ضرب غیر از ضارب است.
    
    هلم جرا: به همین ترتیب بقیه.
  \end{note}
  \vspace*{6cm}
  \begin{note}
    در حرف تصریف نیست چون در آن تصرف نیست. تغییر پذیری در حرف وجود ندارد.
  \end{note}
  \vspace*{3.5cm}
  \begin{note}
    هر اسم و فعلی تغییرپذیر نیست و فقط اسماء متکن یا معرب و افعال متصرف هستند که این حالت را دارند. مثلا فعل «عسی» صرف نمی‌شود، و مضارع ندارد.
  \end{note}
\end{addpage}

% page 5
\begin{addpage}{5}
  \vspace*{3cm}
  \begin{note}
    اسماء مبنی و افعال جامد در علم صرف مورد بحث نیست و در علم نحو مورد بحث قرار می‌گیرد.
  \end{note}
  \begin{note}
    افعال جامد، بعضی‌ها فقط ماضی هستند و بعضی فقط امر هستند. «هب» به معنای «فکر کن، تصور کن». «هب زید شجاعا» به معنای زید را شجاع بدان.
    وهب یهب هب، مشتق است، ولی «هب» به معنای تصور کن از افعال قلوب است و جامد است.
  \end{note}
  \begin{note}
    بعضی از افعال هم تصرف تام ندارند، مثلا مضارع دارند ولی امر ندارند به آن‌ها متصرف ناقص می‌گویند. «ما برح، ما انفک، ما فتئ» به معنای پیوسته در زمان گذشته.
    ما فتئ زید طالبا، زید همیشه دانشجو بود.
  \end{note}
  \begin{note}
    اوشک و کاد در افعال مقاربه فعل مضارع دارند ولی فعل امر ندارند. فقط ماضی و مضارع دارند.
  \end{note}
  \vspace*{3cm}
  \begin{note}
    تصریف افعال: ساخت ماضی و مضارع و امر
  \end{note}
  \begin{note}
    تصریف اسماء: ساخت مفرد و مثنی و جمع و تصغیر و اسم نسبت بسازیم، مثلا شجری، اسم نسبت به شجر است.
  \end{note}
  % end of 1st session
  % start of 2nd session
  \vspace*{1.4cm}
  \begin{note}
    فاعل «استفادت» هی است که به الفاظ بر می‌گردد.
  \end{note}
\end{addpage}

% page 6
\begin{addpage}{6}
  \begin{note}
    فاعل «استفادت» هی است که به الفاظ بر می‌گردد.
  \end{note}
\end{addpage}


\end{document}
