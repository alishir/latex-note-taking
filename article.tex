\documentclass{article}
\usepackage{pdfpages}
\usepackage{environ}
\usepackage[margin=10pt,paperwidth=12in,paperheight=12in]{geometry}
\usepackage[many]{tcolorbox}
\usepackage{xepersian}
\settextfont{Noto Naskh Arabic}
\setlatintextfont{Noto Sans}
\begin{document}

\newenvironment{note}
               {
                 \begin{tcolorbox}[width=0.3\linewidth]
                 \begin{RTL}
               }
               {
                 \end{RTL}
                 \end{tcolorbox}
               }
              
\NewEnviron{addpage}[1]{
  \includepdf[pages=#1,offset=150 0,frame=true,fitpaper=false,pagecommand={
      \BODY
  }]{book.pdf}
}


\includepdf[pages=1,offset=150 0,frame=true,fitpaper=false,pagecommand={
    \vspace*{6cm}
    \begin{note}
        علم العربیة: مضاف و مضاف الیه است و در واقع علم اللغة العربیة بوده و موصوف که اللغة بوده حذف شده است. 
        گاهی اوقات موصوف را در لغت نمی‌آورند.
    \end{note}
    \begin{note}
      صناعة: دانش و فن.
      صوغ: ریختن، اسم فاعل آن صائغ است به معنای ریخته‌گر و اسم مفعول آن مصوغ به معنای ریخته شده.
    \end{note}
    \begin{note}
      خبر «انّ» بر آن مقدم نمی‌شود ولی خبر «کان» مقدم می‌شود.
    \end{note}
    \vspace*{4cm}
    \begin{note}
      نظرات دیگر مانند ۲۸ حرف و ۳۰ حرف هم برای تعداد حروف الفبا وجود دارد.
      \end{note}
}]{book.pdf}

\begin{addpage}{2}
    \vspace*{3cm}
    \begin{note}
      نظرات دیگر مانند ۲۸ حرف و ۳۰ حرف هم برای تعداد حروف الفبا وجود دارد.
     \end{note}
\end{addpage}


\end{document}
